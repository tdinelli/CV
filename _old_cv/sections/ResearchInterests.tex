\section{\sc Research Interests}

% \begin{itemize}
%    \item Developing and maintaining an automatic framework for the
%       collection and analysis of scientific data and kinetic models
%       (\href{https://sciexpem.polimi.it/}{SciExpeM}).
%
%    \item Build reliable pipelines for developing and validating detailed kinetic
%       mechanisms against large numbers of experimental measurements using data science
%       techniques.
%
%    \item Developing automatic routine for optimizing complex kinetic schemes empwering physics-based approaches.
%
%    \item Apply different algorithms for the inverse modeling of high dimensional problems.
% \end{itemize}

% I am a third-year Ph.D. student enrolled at the Politecnico di Milano, affiliated to the
% Department of Chemistry, Materials, and Chemical Engineering Giulio Natta. I am a member
% of the CRECK modeling laboratory, under the guidance of Professor
% Alessandro Stagni. My doctoral research is primarily centered on leveraging data-driven
% methodologies to advance the development of chemical kinetics mechanisms aimed at
% predicting combustion and pyrolysis behaviors of complex fuels. My academic pursuits
% revolve around several key areas of interest:

I am in my third year of Ph.D. at \polimi, affiliated with the Department of
Chemistry, Materials, and Chemical Engineering Giulio Natta. My research takes place
within the CRECK modeling laboratory, under the supervision of Professor Alessandro
Stagni. My Ph.D. work centers on enhancing chemical kinetics models for predicting
combustion and pyrolysis behaviors of complex fuels using data-driven methods.
My academic pursuits revolve around the following areas of interest:

\begin{itemize}
   \item Developing and overseeing an automatic framework (SciExpeM) to systematically
      gather and analyze scientific data and kinetic models.
      \href{https://sciexpem.polimi.it}{Project link}.

   \item Creating robust pipelines to develop and validate detailed kinetic models.
      Models undergoing rigorous testing against extensive experimental datasets using
      data science methodologies.

   \item Designing automated routines to simplify the optimization of complex kinetic
      schemes, thus boosting the effectiveness of physics-based approaches.

   \item Implementing algorithms for the inverse modeling of high-dimensional problems,
      with the goal of improving predictive capabilities in complex systems.
\end{itemize}
